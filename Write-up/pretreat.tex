% matrix: PRE file: pretreat.tex   3 Jul 2014 11:43:29
\begin{table}[htbp]
\caption{\label{tab:pretreat} Comparison of the Pretreatment Variables Across Groups}\medskip
\footnotesize  \begin{center} \begin{tabular}{lcccccc}  \hline \hline    
 \multicolumn{1}{c}{      }  & \multicolumn{3}{c}{Perry Birth Cohort}  & \multicolumn{3}{c}{Abecedarian Birth Cohort} \\  \multicolumn{1}{c}{      }  & \multicolumn{1}{c}{Disadvantaged}  & \multicolumn{1}{c}{Non-Disad.}  & \multicolumn{1}{c}{Non-Disad.}  & \multicolumn{1}{c}{Disadvantaged}  & \multicolumn{1}{c}{Non-Disadv.}  & \multicolumn{1}{c}{Non-Disadv.} \\  \multicolumn{1}{c}{      }  & \multicolumn{1}{c}{Blacks}  & \multicolumn{1}{c}{Blacks}  & \multicolumn{1}{c}{Population}  & \multicolumn{1}{c}{Blacks}  & \multicolumn{1}{c}{Blacks}  & \multicolumn{1}{c}{Population} \\   \hline   
Population  &      751,092 &    3,777,065 &   28,684,543 &      721,922 &      890,328 &   11,484,609 \\[0.1cm]  
Proportion of Total Population &         0.03 &         0.13 &         0.97 &         0.06 &         0.07 &         0.94 \\[0.1cm]  
Mother's years of schooling  &         9.64 &        11.13 &        11.92 &        11.52 &        12.33 &        12.87 \\[0.05cm]  
& (        2.55)& (        2.57)& (        2.44)& (        1.76)& (        1.84)& (        2.07) \\[0.05cm]  
Number of Siblings at Birth  &         5.90 &         4.36 &         3.22 &         1.87 &         1.59 &         1.01 \\[0.05cm]  
& (        3.01)& (        2.87)& (        2.14)& (        2.10)& (        2.53)& (        1.31) \\[0.05cm]  
Family income  &       28,010 &       29,433 &       50,467 &       24,541 &       49,571 &       64,434 \\[0.05cm]  
& (      26,217)& (      30,965)& (      45,671)& (      13,001)& (      20,219)& (      39,442) \\[0.05cm]  
  \hline \hline    \end{tabular}
 \end{center} 
       {\scriptsize  
       {\raggedright 
{\bfseries Notes:} The table shows the means of the variables indicated in the rows for the different groups  indicated in the columns. Standard Deviations are in parenthesis. The NLSY79/PSID weights are used to make each sample representative of its corresponding in the population. Disadvantaged blacks meet the  respective eligibility criteria. (see \citet{Ramey_Smith_1977_AJMD} and \citet{heckman2010analyzing} ). Non-disadvantaged blacks do  not. Non Disadv. Population includes non-disadvantaged blacks and all non-black people.  Family Income is in 2010 dollars. It is measured as a 3-years average around the age of birth on the PSID,   and around age 14-20 in the NLSY. } } 
 \end{table}
